\documentclass[12pt]{article}
\usepackage[spanish]{babel}
\usepackage[utf8]{inputenc}
\usepackage{graphicx}
\usepackage{listings}
\usepackage{hyperref}
\usepackage{xcolor}

\title{Análisis Detallado del Flujo de Información\\Sistema de Chat LCP}
\author{Documentación Técnica}
\date{\today}

\begin{document}
\maketitle

\tableofcontents

\section{Introducción}
Este documento describe detalladamente el flujo de información del sistema de chat LCP (Local Chat Protocol), explicando cada componente, su funcionamiento y las razones detrás de las decisiones de implementación. El análisis incluye una descripción línea por línea de cada proceso importante.

\section{Inicio del Sistema}
\subsection{Punto de Entrada (interfaz.py)}
El sistema comienza su ejecución en \texttt{interfaz.py}. El flujo detallado es:

\begin{enumerate}
    \item Configuración inicial:
    \begin{itemize}
        \item Se importan las bibliotecas necesarias (os, sys, streamlit, datetime)
        \item Se configura el path del proyecto agregando el directorio raíz al sys.path
        \item Se definen constantes críticas:
        \begin{itemize}
            \item \texttt{OFFLINE\_THRESHOLD = 20.0} - Tiempo en segundos para marcar un peer como desconectado
            \item \texttt{MAX\_UPLOAD\_SIZE = 100 * 1024 * 1024} - Límite de 100MB para archivos
            \item \texttt{REFRESH\_INTERVAL = 3000} - Actualización de UI cada 3 segundos
        \end{itemize}
    \end{itemize}

    \item Sistema de autenticación:
    \begin{itemize}
        \item Se verifica si existe \texttt{user\_id} en \texttt{st.session\_state}
        \item Si no existe:
        \begin{itemize}
            \item Se crea un formulario con \texttt{st.form("login\_form")}
            \item Se agrega campo de texto con \texttt{st.text\_input} limitado a 20 caracteres
            \item Al enviar el formulario:
            \begin{itemize}
                \item Se valida que el campo no esté vacío
                \item Se almacena el ID en \texttt{st.session\_state['user\_id']}
                \item Se detiene la ejecución con \texttt{st.stop()} para forzar recarga
            \end{itemize}
        \end{itemize}
    \end{itemize}

    \item Inicialización del motor:
    \begin{itemize}
        \item Se verifica si existe \texttt{engine} en \texttt{st.session\_state}
        \item Si no existe:
        \begin{itemize}
            \item Se crea instancia de \texttt{Engine} con el user\_id
            \item Se llama al método \texttt{start()} para iniciar hilos
            \item Se almacena en \texttt{st.session\_state['engine']}
            \item Si hay error, se muestra con \texttt{st.error} y se detiene
        \end{itemize}
        \item Si existe, se recupera la instancia existente
    \end{itemize}

    \item Configuración de actualización automática:
    \begin{itemize}
        \item Se configura \texttt{st\_autorefresh} con intervalo de 3000ms
        \item Esto garantiza que la UI se mantenga sincronizada con el estado del sistema
    \end{itemize}
\end{enumerate}

\subsection{Panel Lateral de Control}
El panel lateral se construye así:

\begin{enumerate}
    \item Información de usuario:
    \begin{itemize}
        \item Se muestra el ID con \texttt{st.sidebar.title}
        \item Se muestra la IP local con formato HTML y color gris
        \item Se indica el estado TCP con emoji (🟢/🔴)
    \end{itemize}

    \item Búsqueda de peers:
    \begin{itemize}
        \item Se agrega botón "🔍 Buscar Peers"
        \item Al hacer clic:
        \begin{itemize}
            \item Se muestra indicador de estado
            \item Se llama a \texttt{engine.discovery.force\_discover()}
            \item Se muestra confirmación de éxito
        \end{itemize}
    \end{itemize}

    \item Gestión de peers:
    \begin{itemize}
        \item Se obtiene timestamp actual en UTC
        \item Se recuperan peers del discovery con \texttt{get\_peers()}
        \item Para cada peer:
        \begin{itemize}
            \item Se normaliza el ID (elimina padding nulo)
            \item Se convierte entre formatos bytes/string según necesidad
            \item Se mantiene un mapeo inverso para búsquedas rápidas
        \end{itemize}
        \item Se clasifican en actuales/anteriores según \texttt{OFFLINE\_THRESHOLD}
    \end{itemize}
\end{enumerate}

\subsection{Inicialización del Motor (engine.py)}
El motor es el componente central que coordina todos los módulos. El proceso detallado es:

\begin{enumerate}
    \item Constructor de Engine:
    \begin{itemize}
        \item Recibe parámetros:
        \begin{itemize}
            \item \texttt{user\_id}: ID del usuario (string o bytes)
            \item \texttt{broadcast\_interval}: Intervalo de anuncio (default 1.0s)
        \end{itemize}
        \item Normalización del ID:
        \begin{itemize}
            \item Si es string, se convierte a bytes con UTF-8
            \item Se trunca o rellena para tener exactamente 20 bytes
            \item Se usa \texttt{ljust(20, b'\textbackslash x00')} para padding con nulos
        \end{itemize}
    \end{itemize}

    \item Inicialización de componentes:
    \begin{itemize}
        \item \texttt{PeersStore}:
        \begin{itemize}
            \item Se crea instancia sin parámetros
            \item Maneja archivo JSON para persistencia
            \item Se usa para almacenar información de peers
        \end{itemize}
        \item \texttt{HistoryStore}:
        \begin{itemize}
            \item Se crea instancia sin parámetros
            \item También usa archivo JSON
            \item Almacena historial de mensajes y archivos
        \end{itemize}
    \end{itemize}

    \item Configuración del Discovery:
    \begin{itemize}
        \item Se crea instancia con:
        \begin{itemize}
            \item \texttt{user\_id}: ID normalizado
            \item \texttt{broadcast\_interval}: Intervalo configurado
            \item \texttt{peers\_store}: Instancia de PeersStore
        \end{itemize}
        \item Carga de peers previos:
        \begin{itemize}
            \item Llama a \texttt{peers\_store.load()}
            \item Obtiene IP local del discovery
            \item Filtra peers con la misma IP local
            \item Actualiza diccionario \texttt{discovery.peers}
        \end{itemize}
    \end{itemize}

    \item Configuración de Mensajería:
    \begin{itemize}
        \item Se crea instancia con:
        \begin{itemize}
            \item \texttt{user\_id}: ID normalizado
            \item \texttt{discovery}: Instancia de Discovery
            \item \texttt{history\_store}: Instancia de HistoryStore
        \end{itemize}
    \end{itemize}

    \item Método start():
    \begin{itemize}
        \item Crea hilo para \texttt{messaging.recv\_loop}
        \item Configura como daemon (termina con programa principal)
        \item Inicia el hilo para escuchar mensajes
    \end{itemize}
\end{enumerate}

\section{Sistema de Descubrimiento (discovery.py)}
\subsection{Inicialización del Descubrimiento}
El proceso detallado de inicialización incluye:

\begin{enumerate}
    \item Constructor de Discovery:
    \begin{itemize}
        \item Parámetros recibidos:
        \begin{itemize}
            \item \texttt{user\_id}: ID en bytes (20 bytes)
            \item \texttt{broadcast\_interval}: Frecuencia de anuncios
            \item \texttt{peers\_store}: Almacenamiento persistente
        \end{itemize}
        \item Preparación de IDs:
        \begin{itemize}
            \item \texttt{raw\_id = user\_id.rstrip(b'\textbackslash x00')} - Elimina padding
            \item \texttt{user\_id = raw\_id.ljust(20, b'\textbackslash x00')} - Asegura 20 bytes
        \end{itemize}
    \end{itemize}

    \item Detección de IPs:
    \begin{itemize}
        \item Obtiene hostname con \texttt{socket.gethostname()}
        \item Lista todas las IPs con \texttt{socket.gethostbyname\_ex()}
        \item Selección de IP principal:
        \begin{enumerate}
            \item Busca primero IPs que empiecen con "192.168.1."
            \item Si no hay, busca cualquier IP no-loopback
            \item Como último recurso, usa la primera IP disponible
        \end{enumerate}
        \item Registra todas las IPs locales incluyendo 127.0.0.1
    \end{itemize}

    \item Configuración de Socket UDP:
    \begin{itemize}
        \item Crea socket UDP:
        \begin{verbatim}
socket.socket(socket.AF_INET, socket.SOCK_DGRAM)
        \end{verbatim}
        \item Configura opciones:
        \begin{itemize}
            \item SO\_REUSEADDR = 1 (permite reutilizar puerto)
            \item SO\_BROADCAST = 1 (habilita broadcast)
        \end{itemize}
        \item Intenta bind:
        \begin{enumerate}
            \item Primero a IP local seleccionada
            \item Si falla, fallback a "0.0.0.0"
        \end{enumerate}
    \end{itemize}

    \item Inicio de Hilos:
    \begin{itemize}
        \item Broadcast Loop:
        \begin{itemize}
            \item Thread daemon para \texttt{\_broadcast\_loop}
            \item Envía Echo-Request periódicamente
        \end{itemize}
        \item Persist Loop (si hay peers\_store):
        \begin{itemize}
            \item Thread daemon para \texttt{\_persist\_loop}
            \item Guarda estado de peers cada 5 segundos
        \end{itemize}
    \end{itemize}
\end{enumerate}

\section{Sistema de Mensajería (messaging.py)}
\subsection{Inicialización de Mensajería}
El módulo de mensajería implementa la comunicación entre peers. El proceso detallado es:

\begin{enumerate}
    \item Constructor de Messaging:
    \begin{itemize}
        \item Parámetros recibidos:
        \begin{itemize}
            \item \texttt{user\_id}: ID en bytes o string
            \item \texttt{discovery}: Instancia de Discovery
            \item \texttt{history\_store}: Almacenamiento de historial
        \end{itemize}
        \item Normalización del ID:
        \begin{itemize}
            \item Conversión a bytes si es string
            \item \texttt{raw\_id = user\_id.rstrip(b'\textbackslash x00')[:USER\_ID\_SIZE]}
            \item \texttt{user\_id = raw\_id.ljust(USER\_ID\_SIZE, b'\textbackslash x00')}
        \end{itemize}
    \end{itemize}

    \item Configuración de Socket UDP:
    \begin{itemize}
        \item Reutiliza socket del discovery
        \item Configuración de socket:
        \begin{itemize}
            \item \texttt{setblocking(True)} - Operaciones síncronas
            \item \texttt{settimeout(5.0)} - Timeout de 5 segundos
            \item Buffer de recepción: 256KB (\texttt{SO\_RCVBUF = 262144})
            \item Buffer de envío: 256KB (\texttt{SO\_SNDBUF = 262144})
        \end{itemize}
    \end{itemize}

    \item Configuración de Socket TCP:
    \begin{itemize}
        \item Nuevo socket TCP:
        \begin{verbatim}
socket.socket(socket.AF_INET, socket.SOCK_STREAM)
        \end{verbatim}
        \item Configuración:
        \begin{itemize}
            \item Buffers de 256KB (igual que UDP)
            \item Bind a todas las interfaces (0.0.0.0)
            \item Puerto TCP fijo (definido en protocol.py)
            \item Máximo 5 conexiones en cola (\texttt{listen(5)})
        \end{itemize}
    \end{itemize}

    \item Estructuras de Control:
    \begin{itemize}
        \item Sistema de ACKs:
        \begin{itemize}
            \item \texttt{\_acks}: Diccionario de \{uid: threading.Event\}
            \item \texttt{\_acks\_lock}: Lock para sincronización
        \end{itemize}
        \item IDs de mensajes:
        \begin{itemize}
            \item \texttt{\_next\_body\_id}: Contador 0-255
            \item \texttt{\_body\_id\_lock}: Lock para generación de IDs
        \end{itemize}
        \item Headers pendientes:
        \begin{itemize}
            \item \texttt{\_pending\_headers}: Mapeo de \{body\_id: (header, timestamp)\}
            \item \texttt{\_pending\_headers\_lock}: Lock para acceso
        \end{itemize}
        \item Cola de mensajes:
        \begin{itemize}
            \item \texttt{\_message\_queue}: Cola thread-safe
            \item Usada para procesamiento asíncrono
        \end{itemize}
    \end{itemize}

    \item Inicio de Hilos de Mantenimiento:
    \begin{itemize}
        \item Limpieza de headers:
        \begin{itemize}
            \item Thread para \texttt{\_clean\_pending\_headers}
            \item Elimina headers más antiguos de 30 segundos
            \item Se ejecuta cada 5 segundos
        \end{itemize}
        \item Procesamiento de mensajes:
        \begin{itemize}
            \item Thread para \texttt{\_process\_messages}
            \item Consume mensajes de la cola
            \item Maneja errores individualmente
        \end{itemize}
    \end{itemize}
\end{enumerate}

\subsection{Envío de Mensajes}
El proceso detallado de envío de mensajes incluye:

\begin{enumerate}
    \item Método send():
    \begin{itemize}
        \item Parámetros:
        \begin{itemize}
            \item \texttt{recipient}: ID del destinatario (bytes)
            \item \texttt{message}: Contenido del mensaje (bytes)
            \item \texttt{timeout}: Tiempo máximo de espera (default 5s)
        \end{itemize}
    \end{itemize}

    \item Preparación del mensaje:
    \begin{itemize}
        \item Generación de ID:
        \begin{itemize}
            \item Obtiene siguiente \texttt{body\_id} (0-255)
            \item Usa lock para thread-safety
        \end{itemize}
        \item Empaquetado:
        \begin{itemize}
            \item Empaqueta cuerpo con \texttt{pack\_message\_body}
            \item Incluye \texttt{body\_id} y mensaje
        \end{itemize}
    \end{itemize}

    \item Envío del header:
    \begin{itemize}
        \item Construcción:
        \begin{itemize}
            \item \texttt{user\_from}: ID local con padding
            \item \texttt{user\_to}: ID destino con padding
            \item \texttt{op\_code}: OP\_MESSAGE
            \item \texttt{body\_id}: ID generado
            \item \texttt{body\_len}: Longitud del cuerpo
        \end{itemize}
        \item Proceso de envío:
        \begin{itemize}
            \item Llama a \texttt{\_send\_and\_wait}
            \item Espera ACK con timeout
            \item Reintenta hasta 3 veces
        \end{itemize}
    \end{itemize}

    \item Envío del cuerpo:
    \begin{itemize}
        \item Mismo proceso que header:
        \begin{itemize}
            \item \texttt{\_send\_and\_wait} con el cuerpo
            \item Espera ACK con timeout
            \item Reintenta hasta 3 veces
        \end{itemize}
        \item En caso de fallo:
        \begin{itemize}
            \item Intenta redescubrir peers
            \item Propaga excepción si persiste
        \end{itemize}
    \end{itemize}
\end{enumerate}

\subsection{Protocolo de Mensajería}
El protocolo LCP define la estructura de comunicación:

\begin{enumerate}
    \item Header (50 bytes):
    \begin{itemize}
        \item user\_from (20 bytes): ID del remitente
        \item user\_to (20 bytes): ID del destinatario
        \item op\_code (1 byte): Tipo de operación
        \item body\_id (1 byte): ID del mensaje/archivo
        \item body\_len (8 bytes): Tamaño del cuerpo
    \end{itemize}
    \item Respuesta (25 bytes):
    \begin{itemize}
        \item status (1 byte): Código de estado
        \item responder (20 bytes): ID del respondedor
        \item Padding (4 bytes): Alineación
    \end{itemize}
\end{enumerate}

\subsection{Flujo de Mensajes de Texto}
El envío de mensajes sigue este proceso:

\begin{enumerate}
    \item Preparación:
    \begin{itemize}
        \item Genera body\_id único (0-255)
        \item Codifica mensaje en UTF-8
        \item Empaqueta header con op\_code = OP\_MESSAGE
    \end{itemize}
    \item Envío con confirmación:
    \begin{itemize}
        \item Envía header y espera ACK
        \item Envía cuerpo y espera ACK
        \item Reintenta hasta 3 veces con espera exponencial
    \end{itemize}
    \item Procesamiento de recepción:
    \begin{itemize}
        \item Valida header y tamaño
        \item Envía ACK al remitente
        \item Decodifica mensaje y actualiza historial
    \end{itemize}
\end{enumerate}

\subsection{Transferencia de Archivos}
El proceso detallado de transferencia de archivos incluye:

\begin{enumerate}
    \item Método send\_file():
    \begin{itemize}
        \item Parámetros:
        \begin{itemize}
            \item \texttt{recipient}: ID del destinatario (bytes)
            \item \texttt{file\_bytes}: Contenido del archivo
            \item \texttt{filename}: Nombre del archivo
            \item \texttt{timeout}: Tiempo máximo de espera
        \end{itemize}
        \item Validaciones iniciales:
        \begin{itemize}
            \item Verifica existencia del peer en discovery
            \item Obtiene información de IP del peer
        \end{itemize}
    \end{itemize}

    \item Fase UDP (Control):
    \begin{itemize}
        \item Preparación:
        \begin{itemize}
            \item Genera \texttt{body\_id} único
            \item Construye header con \texttt{op\_code = OP\_FILE}
            \item \texttt{body\_len} es el tamaño total del archivo
        \end{itemize}
        \item Envío del header:
        \begin{itemize}
            \item Usa \texttt{\_send\_and\_wait} con timeout
            \item Espera ACK del receptor
            \item Pausa 0.5s para sincronización
        \end{itemize}
    \end{itemize}

    \item Fase TCP (Datos):
    \begin{itemize}
        \item Establecimiento de conexión:
        \begin{itemize}
            \item Crea nuevo socket TCP
            \item Configura buffer de envío (256KB)
            \item Conecta a IP:puerto del receptor
        \end{itemize}
        \item Envío de identificador:
        \begin{itemize}
            \item Convierte \texttt{body\_id} a 8 bytes (big-endian)
            \item Envía bytes del identificador
        \end{itemize}
        \item Transferencia de datos:
        \begin{itemize}
            \item Divide archivo en chunks de 32KB
            \item Para cada chunk:
            \begin{itemize}
                \item Envía datos con \texttt{sock.send}
                \item Verifica bytes enviados
                \item Actualiza contador de progreso
            \end{itemize}
        \end{itemize}
        \item Finalización:
        \begin{itemize}
            \item \texttt{shutdown(SHUT\_WR)} para indicar fin
            \item Espera ACK final (5s timeout)
            \item Verifica estado en respuesta
        \end{itemize}
    \end{itemize}
\end{enumerate}

\subsection{Recepción y Procesamiento}
El sistema de recepción opera en varios niveles:

\begin{enumerate}
    \item Bucle Principal (recv\_loop):
    \begin{itemize}
        \item Inicia thread TCP separado
        \item Bucle infinito para UDP:
        \begin{itemize}
            \item Recibe datos con buffer de 4KB
            \item Valida tamaño mínimo del paquete
            \item Clasifica tipo de mensaje
        \end{itemize}
    \end{itemize}

    \item Procesamiento de ACKs:
    \begin{itemize}
        \item Verifica tamaño (25 bytes)
        \item Desempaqueta con \texttt{unpack\_response}
        \item Si status = 0:
        \begin{itemize}
            \item Busca evento en \texttt{\_acks}
            \item Notifica al sender si corresponde
        \end{itemize}
        \item Pasa respuesta al discovery si no es ACK
    \end{itemize}

    \item Procesamiento de Headers:
    \begin{itemize}
        \item Valida tamaño (50 bytes)
        \item Desempaqueta con \texttt{unpack\_header}
        \item Verifica destinatario:
        \begin{itemize}
            \item Compara con ID local
            \item Verifica si es broadcast
            \item Normaliza IDs eliminando padding
        \end{itemize}
    \end{itemize}

    \item Manejo de Mensajes:
    \begin{itemize}
        \item Para \texttt{OP\_MESSAGE}:
        \begin{itemize}
            \item Envía ACK del header
            \item Espera cuerpo con timeout
            \item Valida tamaño recibido
            \item Envía ACK del cuerpo
            \item Encola para procesamiento
        \end{itemize}
        \item Para \texttt{OP\_FILE}:
        \begin{itemize}
            \item Rechaza si es broadcast
            \item Registra header en \texttt{\_pending\_headers}
            \item Espera conexión TCP
        \end{itemize}
    \end{itemize}

    \item Recepción TCP:
    \begin{itemize}
        \item Función auxiliar \texttt{recv\_exact}:
        \begin{itemize}
            \item Garantiza lectura completa
            \item Acumula datos en bytearray
            \item Maneja desconexiones
        \end{itemize}
        \item Proceso de recepción:
        \begin{itemize}
            \item Lee ID de archivo (8 bytes)
            \item Valida contra headers pendientes
            \item Recibe datos en chunks
            \item Muestra progreso cada 1MB
        \end{itemize}
        \item Post-procesamiento:
        \begin{itemize}
            \item Detecta tipo de archivo
            \item Sanitiza nombre
            \item Guarda en directorio
            \item Actualiza historial
        \end{itemize}
    \end{itemize}
\end{enumerate}

\section{Sistema de Persistencia}
\subsection{Almacenamiento de Peers (peers\_store.py)}
El sistema de almacenamiento de peers opera así:

\begin{enumerate}
    \item Inicialización:
    \begin{itemize}
        \item Constructor:
        \begin{itemize}
            \item Recibe ruta de archivo JSON (opcional)
            \item Crea directorio si no existe
            \item Inicializa archivo vacío si es nuevo
        \end{itemize}
        \item Estructura del archivo:
        \begin{itemize}
            \item Lista de objetos peer
            \item Cada peer tiene:
            \begin{itemize}
                \item ID (key primaria)
                \item IP actual
                \item Último timestamp visto
                \item Estado (conectado/desconectado)
            \end{itemize}
        \end{itemize}
    \end{itemize}

    \item Operaciones de Lectura/Escritura:
    \begin{itemize}
        \item Método load():
        \begin{itemize}
            \item Lee archivo JSON completo
            \item Convierte timestamps a objetos datetime
            \item Normaliza IDs de peers
            \item Retorna diccionario de peers
        \end{itemize}
        \item Método save():
        \begin{itemize}
            \item Recibe diccionario de peers
            \item Convierte timestamps a ISO format
            \item Escribe atomicamente usando archivo temporal
            \item Maneja errores de escritura
        \end{itemize}
    \end{itemize}

    \item Manejo de Concurrencia:
    \begin{itemize}
        \item Lock de archivo:
        \begin{itemize}
            \item Protege acceso al JSON
            \item Previene corrupción
            \item Maneja timeouts
        \end{itemize}
        \item Escritura atómica:
        \begin{itemize}
            \item Escribe a archivo temporal
            \item Renombra al archivo final
            \item Garantiza consistencia
        \end{itemize}
    \end{itemize}
\end{enumerate}

\subsection{Historial de Mensajes (history\_store.py)}
El almacenamiento de historial funciona así:

\begin{enumerate}
    \item Estructura de Datos:
    \begin{itemize}
        \item Archivo JSON principal:
        \begin{itemize}
            \item Lista de entradas de historial
            \item Cada entrada contiene:
            \begin{itemize}
                \item Tipo (mensaje/archivo)
                \item Remitente (ID)
                \item Destinatario (ID o "*global*")
                \item Contenido o nombre de archivo
                \item Timestamp en UTC
                \item Metadatos adicionales
            \end{itemize}
        \end{itemize}
    \end{itemize}

    \item Métodos Principales:
    \begin{itemize}
        \item append\_message():
        \begin{itemize}
            \item Valida parámetros
            \item Normaliza IDs
            \item Agrega timestamp actual
            \item Escribe al historial
        \end{itemize}
        \item append\_file():
        \begin{itemize}
            \item Similar a mensajes
            \item Incluye nombre de archivo
            \item Registra tamaño y tipo
        \end{itemize}
        \item get\_conversation():
        \begin{itemize}
            \item Filtra por participantes
            \item Ordena por timestamp
            \item Excluye mensajes eliminados
        \end{itemize}
    \end{itemize}

    \item Optimizaciones:
    \begin{itemize}
        \item Cache en memoria:
        \begin{itemize}
            \item Mantiene últimas entradas
            \item Reduce accesos a disco
            \item Actualiza periódicamente
        \end{itemize}
        \item Escritura eficiente:
        \begin{itemize}
            \item Append-only por defecto
            \item Buffer de escritura
            \item Flush periódico
        \end{itemize}
    \end{itemize}
\end{enumerate}

\section{Interfaz de Usuario}
\subsection{Componentes Principales}
La interfaz implementa:

\begin{enumerate}
    \item Panel de autenticación:
    \begin{itemize}
        \item Formulario de ingreso de ID
        \item Validación de longitud máxima
        \item Manejo de sesión
    \end{itemize}
    \item Panel lateral:
    \begin{itemize}
        \item Información del usuario
        \item Estado de conexión TCP
        \item Lista de peers activos/inactivos
    \end{itemize}
    \item Área de chat:
    \begin{itemize}
        \item Mensajes globales
        \item Conversaciones privadas
        \item Indicadores de transferencia
    \end{itemize}
\end{enumerate}

\subsection{Actualización Automática}
La interfaz se mantiene sincronizada:

\begin{enumerate}
    \item Refresco cada 3 segundos
    \item Actualización de estado de peers
    \item Monitoreo de transferencias TCP
    \item Indicadores de actividad
\end{enumerate}

\section{Consideraciones de Seguridad}
\subsection{Validaciones Implementadas}
El sistema implementa múltiples capas de seguridad:

\begin{enumerate}
    \item Sanitización de Entradas:
    \begin{itemize}
        \item Nombres de archivo:
        \begin{itemize}
            \item Elimina caracteres no seguros
            \item Limita longitud máxima
            \item Preserva extensión original
            \item Maneja caracteres especiales
        \end{itemize}
        \item IDs de usuario:
        \begin{itemize}
            \item Normaliza a 20 bytes exactos
            \item Elimina caracteres nulos
            \item Valida codificación UTF-8
            \item Previene inyección de bytes
        \end{itemize}
        \item Contenido de mensajes:
        \begin{itemize}
            \item Valida codificación UTF-8
            \item Maneja errores de decodificación
            \item Limita tamaño máximo
            \item Filtra caracteres de control
        \end{itemize}
    \end{itemize}

    \item Protección de Recursos:
    \begin{itemize}
        \item Sistema de archivos:
        \begin{itemize}
            \item Directorio de descargas aislado
            \item Nombres de archivo únicos
            \item Validación de permisos
            \item Límite de espacio en disco
        \end{itemize}
        \item Memoria:
        \begin{itemize}
            \item Límites en buffers
            \item Liberación de recursos
            \item Manejo de fragmentación
            \item Protección contra desbordamiento
        \end{itemize}
        \item Red:
        \begin{itemize}
            \item Timeouts en conexiones
            \item Límites de reintentos
            \item Control de concurrencia
            \item Manejo de desconexiones
        \end{itemize}
    \end{itemize}

    \item Validación de Datos:
    \begin{itemize}
        \item Headers:
        \begin{itemize}
            \item Verifica tamaños exactos
            \item Valida campos obligatorios
            \item Comprueba rangos válidos
            \item Detecta corrupción
        \end{itemize}
        \item Cuerpos de mensajes:
        \begin{itemize}
            \item Valida longitud declarada
            \item Verifica integridad
            \item Detecta truncamiento
            \item Maneja fragmentación
        \end{itemize}
        \item Archivos:
        \begin{itemize}
            \item Valida magic numbers
            \item Detecta tipos MIME
            \item Verifica checksums
            \item Limita tamaños
        \end{itemize}
    \end{itemize}
\end{enumerate}

\end{document} 